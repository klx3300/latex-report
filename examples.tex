\documentclass{article}
    \usepackage[x11names]{xcolor}
\usepackage[T1]{fontenc}
\usepackage{xltxtra}
%\usepackage{xunicode}

%\XeTeXinputencoding "cp936"

\usepackage[top=1.2in,bottom=1.2in,left=1.2in,right=1in]{geometry} % 页边距
\defaultfontfeatures{Mapping=tex-text}%连字号

\usepackage[boldfont,slantfont,CJKchecksingle,CJKnumber]{xeCJK} % 允许斜体和粗体
\xeCJKsetup{PunctStyle=hangmobanjiao}
\setlength{\parindent}{0cm}                 % Default is 15pt.
\linespread{1.2}                       % 行间距
\setlength{\parskip}{\baselineskip}    % 段间距

\XeTeXlinebreaklocale "zh"
\XeTeXlinebreakskip = 0pt plus 1pt minus 0.1pt

% \usepackage[unicode=true,colorlinks,linkcolor=blue]{hyperref} % 超链接
\setCJKmainfont[BoldFont=SimHei, ItalicFont=KaiTi]{SimSun}    %配置中文字体
\setmainfont{Times New Roman}                % 英文衬线字体
\setmonofont{Courier New}                    % 英文等宽字体
\setsansfont{Droid Sans}                     % 英文无衬线字体


\usepackage{graphicx}                   % 嵌入png图像
\usepackage{longtable,tabu,booktabs}
\usepackage{pdflscape}

\usepackage{tocloft}                        % 目录
\renewcommand\contentsname{目录}
\renewcommand{\today}{\number\year 年 \number\month 月 \number\day 日} % 中文日期

\renewcommand\figurename{图} % change figure name
\renewcommand\tablename{表}

\newcommand{\chuhao}{\fontsize{42pt}{\baselineskip}\selectfont}
\newcommand{\xiaochuhao}{\fontsize{36pt}{\baselineskip}\selectfont}
\newcommand{\yihao}{\fontsize{28pt}{\baselineskip}\selectfont}
\newcommand{\erhao}{\fontsize{21pt}{\baselineskip}\selectfont}
\newcommand{\xiaoerhao}{\fontsize{18pt}{\baselineskip}\selectfont}
\newcommand{\sanhao}{\fontsize{15.75pt}{\baselineskip}\selectfont}
\newcommand{\sihao}{\fontsize{14pt}{\baselineskip}\selectfont}
\newcommand{\xiaosihao}{\fontsize{12pt}{\baselineskip}\selectfont}
\newcommand{\wuhao}{\fontsize{10.5pt}{\baselineskip}\selectfont}
\newcommand{\xiaowuhao}{\fontsize{9pt}{\baselineskip}\selectfont}
\newcommand{\liuhao}{\fontsize{7.875pt}{\baselineskip}\selectfont}
\newcommand{\qihao}{\fontsize{5.25pt}{\baselineskip}\selectfont}
    \usepackage{indentfirst}
\usepackage{titlesec}
\usepackage{lastpage}
\usepackage{zhnumber}
\usepackage{amsmath}
\usepackage{caption2}
\renewcommand{\thesubsection}{\arabic{subsection}}
\renewcommand{\thesubsubsection}{\arabic{subsubsection}}
\renewcommand{\captionfont}{\CJKfontspec{SimSun}\fontsize{10.5pt}{\baselineskip}\bfseries}
\renewcommand{\captionlabelfont}{}
\titleformat{\section}
{
    \CJKfontspec{SimHei}\fontsize{14pt}{\baselineskip}\bfseries
}
{
    \zhnumber{\thesection} 、
}
{0pt}
{}
\titleformat{\subsection}
{
    \CJKfontspec{SimHei}\fontsize{14pt}{\baselineskip}\bfseries
}
{
    \thesubsection 、
}
{0pt}
{}
\titleformat{\subsubsection}
{
    \CJKfontspec{SimSun}\fontsize{12pt}{\baselineskip}\bfseries
}
{
    (\thesubsubsection) 
}
{0pt}
{}
\newenvironment{RepTitAll} % overall title
{% before
    \CJKfontspec{SimHei}
    \centering
    \xiaochuhao
}
{%after

}
\newenvironment{ScoreTableTitle} % self-explain
{
    \CJKfontspec{SimHei}
    \centering
    \erhao
}
{

}
\newenvironment{SignPlace}
{
    \CJKfontspec{SimSun}
    \sihao
}
{

}
\newenvironment{PartTitle} % Part A/B/C/D... Title
{
    \CJKfontspec{FZShuTi}
    \erhao
}
{

}
\newenvironment{MainBody} % The main content
{
    \setlength{\parindent}{2em}
    \CJKfontspec{SimSun}
    \wuhao
}
{

}
\newpagestyle{CoverPage}
{
    \setfoot{}{\thepage/\pageref{LastPage}}{}
    \headrule
}
\newpagestyle{ContentPage}
{
    \sethead{}{\sihao {\textbf{《数字电路与逻辑设计》实验报告}}}{}
    \setfoot{}{\thepage/\pageref{LastPage}}{}
    \headrule
}

% these stuff controls the figure and table numbering.
% if you are writing books(include chaps.), uncomment comments below.
% otherwise, keep them commented.

\makeatletter
\renewcommand \thefigure  { %\ifnum \c@chapter>\z@ \thechapter-\fi
\ifnum \c@section>\z@ \@arabic\c@section-\fi \ifnum
\c@subsection>\z@ \@arabic\c@subsection-\fi\ifnum
\c@subsubsection>\z@
\@arabic\c@subsubsection-\fi\@arabic\c@figure}
\makeatother

\makeatletter
% \@addtoreset{subsubsection}{chapter}
\@addtoreset{subsubsection}{section}
% \@addtoreset{figure}{chapter}
\@addtoreset{figure}{section}
\@addtoreset{figure}{subsection}
\@addtoreset{figure}{subsubsection}
\makeatother

\makeatletter
\renewcommand \thetable  { %\ifnum \c@chapter>\z@ \thechapter-\fi
\ifnum \c@section>\z@ \@arabic\c@section-\fi \ifnum
\c@subsection>\z@ \@arabic\c@subsection-\fi\ifnum
\c@subsubsection>\z@
\@arabic\c@subsubsection-\fi\@arabic\c@table}
\makeatother

\makeatletter
% \@addtoreset{subsubsection}{chapter}
\@addtoreset{subsubsection}{section}
% \@addtoreset{table}{chapter}
\@addtoreset{table}{section}
\@addtoreset{table}{subsection}
\@addtoreset{table}{subsubsection}
\makeatother
    \setcounter{tocdepth}{3}
    \usepackage{amsmath}
    \usepackage{subcaption}
    \usepackage{multirow}
    \title{测试文档}
    \date{2018-09-01}
    \author{张三}
    \begin{document}
        \pagenumbering{gobble}
        \maketitle
        \newpage
        \tableofcontents
        \newpage
        \pagenumbering{arabic}
        \section{中文段落}
        你好,世界!
        \subsection{中文子段落}
        测试子段落内容。
        \paragraph{中文段落2}
        段落内容。
        \subparagraph{中文子段落2}
        段落内容2。\\
        换行测试。
        \paragraph{公式}
        \begin{equation}
            f(x) = x^2
        \end{equation}
        \paragraph{无编号公式}
        \begin{equation*}
            f(x) = x^3
        \end{equation*}
        \paragraph{内联公式}
        我们需要的结果由公式 $f(x)=sin(x)$ 计算得到。
        \paragraph{公式按等号对齐}
        必须使用  将公式分开,否则同一环境下无法写两个公式
        \begin{align*}
            x + y &= z\\
            x &= z - y
        \end{align*}
        \paragraph{矩阵}
        \begin{equation}
            \left[
            \begin{matrix}
                1 & 0 & 1\\
                0 & 1 & 0\\
                1 & 0 & 1
            \end{matrix}
            \right]
        \end{equation}
        \paragraph{扩充型括号}
        $\left(\frac{1}{\sqrt{x}}\right)$
        \paragraph{图片}
        \begin{figure}
            \includegraphics[width=\linewidth]{myavatar.png}
            \caption{我的头像}
            \label{fig:myava1}
        \end{figure}
        图 \ref{fig:myava1} 是我的头像。
        \paragraph{二联图片}
        \begin{figure}[h!]
            \centering
            \begin{subfigure}[b]{0.4\linewidth}
                \includegraphics[width=\linewidth]{myavatar.png}
                \caption{我的头像}
            \end{subfigure}
            \begin{subfigure}[b]{0.4\linewidth}
                \includegraphics[width=\linewidth]{myavatar.png}
                \caption{我的头像的复制}
            \end{subfigure}
            \caption{我的头像二联}
            \label{fig:linkedavatar}
        \end{figure}
        \paragraph{表}
        \begin{table}[h!]
            \begin{center}
                \caption{示例表}
                \label{table:extab1}
                \begin{tabular}{l|c|r} % first left, second center, third right.
                    \hline
                    \textbf{数据甲} & \textbf{数据乙} & \textbf{数据丙}\\
                    \hline
                    1 & 1.2 & A\\
                    2 & 2.3 & B\\
                    3 & 3.4 & C\\
                    \hline
                    \multirow{2}{*}{spanned} & 4.5 & E\\
                    & 5.6 & F\\
                \end{tabular}
            \end{center}
        \end{table}
        \section{正文测试}
        \begin{MainBody}
            \indent 燕子去了,有再来的时候;杨柳枯了,有再青的时候;桃花谢了,有再开的时候。但是,聪明的,你告诉我,我们的日子为什么一去不复返呢?——是有人偷了他们罢:那是谁?又藏在何处呢?是他们自己逃走了罢:现在又到了哪里呢?    \\
            \indent 我不知道他们给了我多少日子;但我的手确乎是渐渐空虚了。在默默里算着,八千多日子已经从我手中溜去;像针尖上一滴水滴在大海里,我的日子滴在时间的流里,没有声音,也没有影子。我不禁头涔涔而泪潸潸了。\\
        \end{MainBody}
        \newpage
        \pagenumbering{arabic}
        \pagestyle{CoverPage}
        \begin{center}
            \includegraphics{hust.png}
        \end{center}
        \begin{RepTitAll}
            数字逻辑实验报告(1)
        \end{RepTitAll}
        \begin{center}
            \CJKfontspec{SimHei}
            \erhao
            \begin{tabular}{|c|c|c|}
                \hline
                \multicolumn{3}{| >{\centering}p{\linewidth}|}{数字逻辑实验 1}\\
                \hline
                \sihao
                系列二进制加法器设计 50\% & \sihao 小型实验室门禁系统设计 50\% & \sihao 总成绩 \\
                \hline
                 & & \\
                 & & \\
                \hline
            \end{tabular}
        \end{center}
        \begin{center}
            \xiaosihao
            \begin{tabular}{|p{\linewidth}|}
                \hline
                评语:\\
                \\
                \\
                \\
                \\
                \\
                \\
                \\
                \\
                {
                    \begin{verbatim}
                                        教师签名
                    \end{verbatim}
                }\\
                \hline
            \end{tabular}
        \end{center}
        \begin{center}
            \begin{SignPlace}
                \begin{tabular}{ccccc}
                    \textbf{姓} & & & \textbf{名}&\textbf{:}$\rule{4cm}{0.30mm}$\\
                    \textbf{学} & & & \textbf{号}&\textbf{:}$\rule{4cm}{0.30mm}$\\
                    \textbf{班} & & & \textbf{级}&\textbf{:}$\rule{4cm}{0.30mm}$\\
                    \textbf{指}&\textbf{导}&\textbf{教}&\textbf{师}&\textbf{:}$\rule{4cm}{0.30mm}$\\
                \end{tabular}
            \end{SignPlace}
        \end{center}
        \begin{center}
            \begin{SignPlace}
                \textbf{计算机科学与技术学院}
            \end{SignPlace}
        \end{center}
        \newpage
        \pagestyle{ContentPage}
        \section{Test}
        \subsection{TestII}
        \subsubsection{TestIII}
        \newpage
        \begin{appendix}
            \listoffigures
            \listoftables
        \end{appendix}
    \end{document}